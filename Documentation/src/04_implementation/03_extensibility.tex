\chapter{Extensibility of the Substitution Stepper}
A secondary goal when creating the Substitution Stepper was to create a stepping application that can be used for more than just Haskell.
For this, the stepping functionality was kept as generic as possible.
This chapter describes how one could approach the addition of a new language that should be stepped.

\section{Implementing the Show/Pretty Class}
The Substitution Stepper requires the terms to be pretty printable via the \texttt{pretty} function defined in the Prettyprinter package,
as well as transformable into strings via show.

The requirement for Show could be lifted,
but that would require removing the showTerm command.
That should however not have too big of an impact on the usability.

\section{Implementing the Term Class}
The first step when adding a new language to step,
should be to implement the Term Class,
as it is used everywhere across the code.

When adding a structure that keeps track of annotations,
it is important that the annotation does not change the position of the contained terms.
Other than that there is not much to it and implementing the Term class should be quite simple.

\section{Implementing the Reducible Class}
The Reducible class needs to be implemented,
in order to be able to step the term.

\section{Optional: Implementing the Normalizable Class}
The Normalizable class can be useful if the derivation of a term should hide some steps that are not really interesting.
But it is not necessary to do that.
If this functionality is not desired,
it might be best to implement \texttt{getNormalizableTerm},
but rather than performing some changes to the term,
it can be wrapped in a Just and returned.
Alternatively,
the getNormalizableTerm function can also just always return Nothing.

\section{Optional: Implementing Highlightable}
The Highlightable is also not essential to the derivation.
However, it is used to color the selection and diff during the manual derivation.
If only the automatic derivation functionality is desired,
then the Highlightable class can be implemented with bogus functions,
like \texttt{annotateTerm = \\\_ -> id}, \texttt{combineAnnotations = \\\_ -> id},
\texttt{highlight = pretty}, and \texttt{unAnnotateTerm = id}.

\section{Setting up the Derivation}
Once the above classes have been implemented,
the derivation can be set up and then executed.

At first, if any type of delta reduction is supported,
the rules need to be created.
The rules should have the following type: \texttt{type RewriteRule t = (Subterm t -> Maybe (t, RuleDescription))}.

Once all rules are created,
the functions that execute the derivation can be called.
These function are: \texttt{doAutomaticDerivation},
\texttt{doManualDerivation},
and \texttt{doInteractiveDerivation}.

When all of this is done,
the new language is ready to be stepped.

\section{Conclusion}
It is fairly simple to add different languages to the Substitution Stepper.
The most challenging task is probably to implement the Reducible class and maybe create the RewriteRules, which depends heavily on the language added.
The other parts are quite simple.

Since there is not as much code to be written for the addition of a new language,
a lot of time can be saved if a similar project is attempted again.

Thus, I would say that the secondary goal of being extendible has been reached.
