\chapter{Quality Measures}

This chapter describes the various quality measures that were put in place to make sure that the code is as clean as possible and working as expected.

\section{Coding Guidelines}

\begin{itemize}
    \item General
    \begin{itemize}
        \item Code needs to be committed frequently with descriptive commit messages.
        \item Side effects should be avoided whenever possible.
        \item Comments may only be used if they are helpful.
        \item Global variables should be avoided if possible.
        \item Exceptions are preferred over error codes.
        \item Nesting should not be deeper than 2 levels.
        \item No duplicate code. (DRY)
        \item Functions should be kept short (less than 11 lines).
        \item Create feature branches, before every merge, tests must be run.
    \end{itemize}
    \item Formatting
    \begin{itemize}
        \item Line length should not exceed 120 characters.
        \item Style guide: ESLint + AirBnB
        \item Code must be auto-formatted before being committed.
        \item Code must pass all tests before being committed.
    \end{itemize}
    \item Naming
    \begin{itemize}
        \item Function names should be verbs.
        \item Class names should be nouns.
        \item Longer, more descriptive names are preferred over short ones.
        \item No abbreviations in names.
        \item No 1-character names.
    \end{itemize}
\end{itemize}

\section{Test Concept}

\section{Workflow}

Pipelines?
