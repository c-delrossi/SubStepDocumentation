\chapter{Quality Measures}

This chapter describes the various quality measures that were put in place to make sure that the code is as clean as possible and working as expected.

\section{Coding Guidelines}

\begin{itemize}
    \item General
    \begin{itemize}
        \item Code needs to be committed frequently with descriptive commit messages.
        \item Side effects should be avoided whenever possible.
        \item Comments may only be used if they are helpful.
        \item Global variables should be avoided if possible.
        \item Exceptions are preferred over error codes.
        \item Nesting should not be deeper than 2 levels.
        \item No duplicate code. (DRY)
        \item Functions should be kept short (less than 11 lines).
    \end{itemize}
    \item Formatting
    \begin{itemize}
        \item Line length should not exceed 120 characters.
    \end{itemize}
\end{itemize}

\section{Test Concept}

Unfortunately, a proper test concept was cut from the final product due to the limited amount of time available.
However, during the development, some manual testing was performed.
During the refactoring and extending of the functionality,
the results of the automatic derivations were compared to the expected result and thus it could be verified that the changes did not break the stepping.

The creation of test cases was further hindered due to the fact that the creation of \texttt{CoreExprs} is very cumbersome,
as there is no simple constructor for that.

