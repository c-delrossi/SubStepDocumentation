\chapter{Requirements}

\section{Functional Requirements}

The following table lists the identified functional requirements.
The MVP includes FR1, FR3, FR4, FR5, and FR6.

\begin{table}[ht!]
    \makebox[\textwidth][c]{
        \begin{tabular}{ p{1.5cm} p{12.65cm}}
            \textbf{FR} & \textbf{Description} \\ \hline \\                                                                                                                                                                              
            \textbf{FR 1} & The user can specify functions that should not be stepped through and instead derived in a single step. \\
            \textbf{FR 2} & The user can control the derivation flow in such a way, that they can skip the derivation for certain redexes interactively. \\
            \textbf{FR 3} & The user can step through the reduction line by line and control which subterm should be reduced next. \\
            \textbf{FR 4} & The user can make the derivation run through to the end without requiring any interaction. \\
            \textbf{FR 5} & The tool supports a verbose variant, that indicates which function has been applied. \\
            \textbf{FR 6} & The user can import user-defined files and step through functions defined in these files. \\
        \end{tabular}
    }
    \label{tab:functionalRequirements}
\end{table}

\section {Non-Functional Requirements}

This section contains the NFRs, which are prioritized from 1 to 3, where 1 is the highest priority and 3 is the lowest priority.

\begin{table}[ht!]
    \makebox[\textwidth][c]{
        \begin{tabular}{ |p{1.3cm} |p{7cm} |p{7cm} |p{0.85cm}|}
            \hline
            \textbf{Nr.} & \textbf{Description} & \textbf{Measurement}  & \textbf{Prio} \\ \hline                                                                        
            NFR1         & The tool is usable for people with little experience in functional programming.  & The derivation of a user-defined function takes max. one import command and one step command. The function is defined in one file and depends only on other functions in the same file. &  1 \\ \hline
            NFR2         & The tool provides an open API or CLI, to open it up to additional UIs. & & 3 \\
            \hline
        \end{tabular}
    }
    \label{tab:NFRs}
\end{table}
