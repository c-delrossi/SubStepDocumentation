\newpage

\section*{Glossary}

\begin{description}
    \item[Alternative] one of the subterms that the scrutinee is matched against.
        An alternative defines a pattern that the scrutinee is being matched against
        and a replacement for the scrutinee, in case the scrutinee matches the pattern
    \item[Binding] The combination of a function- or variable name and the definition.
        The definition is 'bound' to the name/identifier.
    \item[Core] An intermediate language that Haskell is compiled to.
        It is a simplified language compared to core but it can do anything that Haskell can do too.
    \item[Derivation] A step-by-step reduction of a term, often resulting in normal form
    \item[Desugaring] The process of removing certain Haskell features to create a simpler term to derive.
        Can be thought of removal of 'syntactic sugar'.
    \item[LHS] Left-Hand-Side 
    \item[RHS] Right-Hand-Side
    \item[Redex] A (sub)term that is reducible. I.e. a reduction rule can be applied to the term successfully.
    \item[Reduction] A rule-based step applied to a term that changes the structure of that term
    \item[Scrutinee] A term examined in a case expression
    \item[Sharing] A mechanism used by Haskell to prevent re-evaluation of the same term.
    \item[WHNF] \href{https://wiki.haskell.org/Weak_head_normal_form#:~:text=An%20expression%20is%20in%20weak,abstraction%20%5Cx%20%2D%3E%20expression%20.}{Weak Head Normal Form}
\end{description}