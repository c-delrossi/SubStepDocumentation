\chapter{Conclusion \& Next Steps}

\section{Requirements}

In the previous chapter \ref*{chptr:Results},
the requirements were analyzed and checked whether they have been met or not.
As table \ref*{tab:RequirementSummary} shows,
most of the requirements were met.

FR1 probably requires an approach different from what has been attempted
and FR6 could probably be implemented without too much of a struggle.
Since the CoreStepper is already capable of loading a module and returning all the bindings in there,
it is reasonable to assume that one could use that multiple times on different modules to get all bindings defined across all the modules.

\begin{table}[ht!]
    \makebox[\textwidth][c]{
        \begin{tabular}{ |p{3cm} |p{3cm} |}
            \hline
            \textbf{Requirement} & \textbf{Status} \\ \hline
            \color{BrickRed} FR  1 & \color{BrickRed} Not met \\ \hline
            \color{Green}    FR  2 & \color{Green}    Met \\ \hline
            \color{Green}    FR  3 & \color{Green}    Met \\ \hline
            \color{Green}    FR  4 & \color{Green}    Met \\ \hline
            \color{Green}    FR  5 & \color{Green}    Met \\ \hline
            \color{Orange}   FR  6 & \color{Orange}   Partially met \\ \hline
            \color{Green}    NFR 1 & \color{Green}    Met \\ \hline
            \color{Green}    NFR 2 & \color{Green}    Met \\ \hline
        \end{tabular}
    }
    \caption{Summary of which requirements have been met and which have not.}
    \label{tab:RequirementSummary}
\end{table}

Originally the MVP was planned to contain FR1, FR3, FR4, FR5, and FR6.
Despite FR1 not being met, I would consider the MVP being reached.
I would say that FR1 is not as important as FR2.
So I would redefine the MVP to contain FR2, FR3, FR4, FR5, and FR6.


\ \\
This has a couple of reasons.
First of all,
the two requirements are very similar.
In essence, they have the same responsibility.
Both requirements include functionality that happens in the background.
However, the functionality described by FR1 does not happen automatically.
The user needs to manually add functions that should be skipped.
This makes the usage of FR1 more complicated.

On the other hand, in order to make use of the functionality described in FR2 the user does not have to do anything.
For both the interactive and manual derivation modes,
the skipping of uninteresting steps is the default behavior.
This, in my opinion, is already a big advantage over the functionality described in FR1.


\ \\
Secondly,
when the requirements for the Substitution Stepper were created,
FR1 was counted towards the MVP because it seemed easier to implement than FR2,
which lead to it being prioritized more.
During the implementation,
 it became clear that FR2 would be easier to implement than FR1,
which is why the priorities have shifted.

\ \\
For these reasons, I have decided to implement FR2 before FR1,
even though originally FR1 was planned to be part of the MVP, not FR2.

\section{Reflection}
The goal of this thesis was to improve the usability of the Substitution Stepper.
In the end,
the thesis was not only limited to usability but included also a bit of a redesign of the stepping functionality.
Due to change and expansion of scope,
some requirements that were originally created could not be met.

\ \\
During this thesis,
advances, like the ability to step instance functions, were made,
thus reducing the limitations of the Substitution Stepper.

The redesign of the stepping functionality also came with the advantage of being theoretically able to step other languages that are not Haskell,
like the Lambda Calculus.
It is possible now to add new steppable languages with relatively low effort,
since the user interaction does not depend on the stepped language.
Things like the highlighting of subterms and selection of subterms to derive don't have to be implemented for new languages.
This can greatly reduce the effort that is needed to execute a project similar to this in the future.

\ \\
Despite not all the requirements being met,
the result of the thesis is a usable application that allows the user to step through Haskell code.
The most important requirements for a working product were met and I am confident that the application would prove useful to someone learning Haskell.
For this reason I would say that the main goal of the thesis was reached.
The following section will provide some ideas for the next steps that could further improve the Substitution Stepper.

\ \\
While I consider this thesis a success,
there are still some things that I would have adjusted in retrospect.

\ \\
For starters,
I would probably adjust the \texttt{Subterm} data structure a bit.
The way annotations are handled currently is a bit more complicated than it needs to be.
Instead of making the stepped terms handle annotations themselves,
I would make the \texttt{Subterm} data structure handle the annotations.

By doing this,
the \texttt{Highlightable} class would not have to be implemented by every single term structure that is being stepped.
It could be implemented using only the \texttt{Subterm} data structure.
Additionally,
one would not have to worry about the annotations changing the actual position of the subterm,
which is something that could happen if the annotations are not implemented properly.

In short: Making the annotations part of the \texttt{Subterm} could help reduce code duplication, and in general,
the amount of code that needs to be written to add another term structure to the Substitution Stepper.

\ \\
Another change I would do is to add some kind of UI interface that allows the modules responsible for the derivation to be decoupled from the UI modules.
This change is described in subsection \ref*{subsect:UIInterface}.


\section{Next Steps}

This section proposes adjustments or expansions of the Substitution Stepper that could be done when continuing development on this project.
The adjustments/expansions are not listed in any particular order.

\subsection{Support for the Prelude}
One of the probably most impactful adjustments would be to add support for the Prelude.
Since the limitation of not being able to use the data types defined in the prelude is quite a big handicap,
it would make sense to work on this first.

Considering that especially the numerical data types are used often in the Functional Programming class to demonstrate functions like sum, foldl/foldr, etc.
it would be very nice for the students using this tool during class to be able to use them.
Another important concept of Haskell, which is currying, could not be demonstrated either,
since tuples are not supported either.
And from my experience, functions like (un)curry or flip were the functions that most students struggled with a bit, besides Monads and Applicatives.

Having to implement your own data type is a bit of a big hurdle at the start.
Especially data types like \texttt{data Nat = S Nat | Z} are hard to understand when starting out with functional programming.

\ \\
For these reasons,
I would strongly recommend making this a priority when continuing development on the Substitution Stepper.


\subsection{Collapsing/Shortening of (Sub)terms}
Collapsing or shortening (sub)terms was originally also a desired functionality.
However, it did not make the list of requirements as it was not prioritized highly enough.
During the development of the Substitution Stepper there was not enough time to add additional requirements,
which is why it did not make the list later either.

Despite this,
being able to shorten a term would be very useful,
especially since some terms can get very long.

\ \\
An idea for how to implement this was to let the user manually choose a subterm and offering them a command with which they could replace the whole subterm with a custom name.
This would be somewhat simple to implement as it does not require any 'intelligence' on the side of the application.
In an earlier iteration of the Substitution Stepper the \texttt{mode} command was used to switch between the selection of redexes only or the selection of any arbitrary subterm.
It would be imaginable to bring back this functionality and to add a \texttt{rename} or \texttt{replace} command that would replace the selected subterm with the specified name.

This could be a way to make the term shorter,
however it might not be an ideal approach as it requires additional user interaction,
which might be a bit cumbersome for the user.

Another problem with this kind of approach is that in long terms there is an exponentially larger amount of subterms,
which could make the selection of subterms even more cumbersome if no better way is found besides using the tab key.

\ \\
The collapsing or shortening of (sub)terms could maybe also be done automatically,
however,
I have not been able to come up with an idea on how to do that.

\subsection{Simplification of the UI modules}
\label{subsect:UIInterface}
To simplify code and prevent duplication,
maybe a \texttt{UI} class could be created that is used in the \texttt{Derivation.Manual} and \texttt{Derivation.Interactive} modules.
This could allow these modules to call UI functions while still remaining independent of the \texttt{ManualUI} and \texttt{InteractiveUI} modules.
This would probably be better practice,
as it would also make it easier to replace the UI,
since the \texttt{Derivation.Manual} and \texttt{Derivation.Interactive} modules could be left unchanged.

I have tried to create a UI class that acts as an interface between the UI modules and the Derivation modules.
I might have taken the wrong approach there,
as I was not able to implement it in a sufficiently elegant way.

\subsection{UI State}
\label{subsect:UIState}
Adding something like a UI State to the UI types might also prove beneficial.
The UI State could for example keep track of how many lines have been written,
in order to be able to erase them safely and reliably.

The current implementation is not very elegant,
as it re-computes the length of the previous prompt and then erases that amount of lines.
This could potentially be an error-prone approach,
while the approach using the UI State is probably safe.

I could also imagine the UI State keeping track of the current CompletionFunc.
This might allow for the usage of multiple completion functions during runtime.
For example, if no function is specified in the command line arguments,
the application could prompt the user to input the function name,
and provide completion support for the function names when doing so.

And as soon as the derivation process starts,
the application could switch the completion function and provide the usual completion function that cycles the selection of redexes.

\ \\
This change could probably be done in one go together with the simplification of the UI modules,
improving the overall cleanliness of the UI code.

\subsection{Improvements to Pretty Printing}
There are still some improvements that could be made to the pretty printing of the Core expressions.
Especially when it comes to operator precedence and the placement of line breaks in the CLI.

The Substitution Stepper currently disregards operator precedence completely as I have not gotten to look into that yet.
The placement of line breaks is not perfected either,
as sometimes it can look a bit confusing when the terms get too long.

Overall, this would be an improvement that is worth implementing.
It would make it easier for users to read and understand what is going on.

\subsection{Printing Terms via Show}
During the development of the Substitution Stepper there was an idea that could help the readability of the terms greatly.
The idea was to display data types that implement the Show class via show, instead of showing the constructors.
This however was not possible due to the limitation that the Prelude - and thus strings - are not supported currently.

But even if this limitation is resolved,
it might still not be possible to implement this idea.
To be able to implement this idea,
the Substitution Stepper needs to know the type of an expression before attempting to display it via show.
Since the Substitution Stepper currently ignores types,
some changes might be needed to enable this.
If the Show instance is constrained to f.ex. another type needing to implement Show as well,
the implementation of this feature might become very complex.

\subsection{Implementation of different Languages}
It might be interesting to implement additional languages that can be derived by the Substitution Stepper.
For example,
the Lambda Calculus language might be interesting for the Functional Programming class,
as it is also part of the course.
Considering that there is already a LambdaCalculator project used during the course,
this might not be necessary.
Though I could imagine that it might be convenient to have the capability to step Haskell and Lambda Calculus in the same application.
