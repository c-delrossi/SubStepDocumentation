\chapter{Conclusion \& Next Steps}

\instructions{
    Ergebnisdiskussion: In der Schlussfolgerung werden die Ergebnisse reflektiert und von Ihnen bewertet.
    Somit wird die Zielerreichung gemessen (Abgleich mit "Aufgabenstellung" und "Ziel der Arbeit") und ein
    Vergleich mit anderen/vorherigen Lösungen hergestellt. Die Schlussfolgerungen bilden einen wichtigen
    Abschnitt eines Berichts und sollen daher sorgfältig ausgearbeitet sein.

    Zudem erfolgt ein Ausblick auf mögliche Weiterentwicklungen, allfällige Verbesserungen oder neue
    Fragenstellungen, die sich aus Ihrer Arbeit ergeben.
    Fragen, die Sie sich selbst stellen können:
    \begin{itemize}
        \item Was wurde (Neues) erreicht?
        \item Was wurde nicht oder noch nicht genügend gut erreicht?
        \item  Was bleibt noch zu tun?
        \item  Welche neuen Fragestellungen ergeben sich aus Ihrer Arbeit?
    \end{itemize}
}

\section{Functional Requirements}

\subsection{Non-Functional Requirements}

\section{Reflection}

\section{Next Steps}
