\chapter{Results}
    \label{chptr:Results}

In this chapter,
the final result is analyzed and the requirements are checked.

\instructions{
    Ergebnisse der Arbeit: Was wurde erreicht, was wurde nicht erreicht? Stellen Sie einen konkreten Bezug zu
    den Anforderungen (FA, NFA) her und verknüpfen Sie diese mit Ihren Ergebnissen. Bleiben Sie objektiv und
    nehmen Sie (noch) keine Wertung Ihrer Arbeit vor (siehe "Schlussfolgerungen").
    Ergebnisse der Arbeit: Was wurde erreicht, was wurde nicht erreicht? Stellen Sie einen konkreten Bezug zu
    den Anforderungen (FA, NFA) her und verknüpfen Sie diese mit Ihren Ergebnissen. Bleiben Sie objektiv und
    nehmen Sie (noch) keine Wertung Ihrer Arbeit vor (siehe "Schlussfolgerungen").
}

\section{Functional Requirements}

\subsection{FR 1}
The user can specify functions that should not be stepped through and instead derived in a single step.

\ \\
The first requirement has not been met,
as I have not found a possible way to reliably implement this functionality.
This requirement is closely related to normalization,
as the point is to skip all applications of a certain function until it does not contain that function anymore.

An example that was created as a possible use case for this requirement was the \texttt{+} operator.

\begin{figure}[!ht]
\begin{verbatim}
    data Nat = S Nat | Z

    (+) :: Nat -> Nat -> Nat
    Z + n = n
    S m + n = m + S n

    ex = (S (S (S Z))) + (S Z)
\end{verbatim}
    \caption{The starting term for the example of FR 1}
    \label{fig:FR1example}
\end{figure}

The idea here was that since the \texttt{+} operator is very trivial and boring to step through due to the big amount of recursion just to add two numbers together,
that it should be skipped instead and directly return the result (\texttt{S (S (S (S Z)))} in this case).

However, due to the recursive nature of the definition of the \texttt{+} operator,
I found this to be not possible to do nicely.

While it would work for this particular example,
it would not work for a term that does not have a normal form,
since then the \texttt{+} operator could end up being expanded infinitely.

\ \\
Looking at the term \texttt{pure + <*> [S Z] [S (S Z)]} shows the problem in this case.
If the function \texttt{pure} is reduced,
the Substitution Stepper will eventually get to the \texttt{+} inside of it and try to skip it.
But because it is not applied to enough arguments,
it does not have a normal form.
Instead, the \texttt{+} will be expanded over and over again, causing an endless loop (see figure \ref*{fig:FR1infiniteLoop}).

\begin{figure}[!ht]
\begin{verbatim}
    pure +

    [+]

    [
        \left right -> case left of
            Z -> right
            S m -> m + S right
    ]

    [
        \left right -> case left of
            Z -> right
            S m -> case m of
                Z -> S right
                S m' -> m' + S (S right)
    ]
    ...
\end{verbatim}
    \caption{Endless recursion caused due to the term \texttt{pure +} not having a normal form}
    \label{fig:FR1infiniteLoop}
\end{figure}

\newpage
\subsection{FR2}
The user can control the derivation flow in such a way, that they can skip the derivation for certain redexes interactively.

\ \\
This requirement has been met.
It corresponds to the call command.
Using this command the user can skip case applications and beta reductions.
This allows the user to ignore the mostly uninteresting parts of the derivation.

Since there are not many cases where a user would want to see all the detailed steps including beta reductions and case application,
this setting has been made the default setting for the interactive derivation modes.

In addition to the call command, there is also a command named normalize,
which is taking the skipping of steps even further.
Due to the recursive nature of the normalize command, however,
it is not used as a default command anywhere, as it could easily cause an infinite loop if applied to a term that does not have a normal form,
similarly to the problem described in the first FR.

The following two examples from the Substitution Stepper show the difference between using the call command (figure \ref*{fig:FR2exampleCall}) and the regular step command (figure \ref*{fig:FR2exampleStep}).

\begin{figure}[!ht]
\begin{verbatim}
$ stack exec -- corestepper-exe -m manual -F ../test/Stepped.hs -f test6
(S (S Z)) (+) S (S Z)


Please enter a command:
{- Normalize (((\ds_d2Mn n_a20X -> case ds_d2Mn of
    Z -> n_a20X;
    (S m_a20Y) -> m_a20Y + S n_a20X
)) (S (S Z))) (S (S Z)) -}


(S Z) (+) S (S (S Z))
\end{verbatim}
\caption{Derivation using the call command}
\label{fig:FR2exampleCall}
\end{figure}

\begin{figure}[!ht]
\begin{verbatim}
$ stack exec -- corestepper-exe -m manual -F ../test/Stepped.hs -f test6
(S (S Z)) (+) S (S Z)


Please enter a command:
{- DeltaReduction + -}

(((\ds_d2Mn n_a20X -> case ds_d2Mn of
    Z -> n_a20X;
    (S m_a20Y) -> m_a20Y + S n_a20X
)) (S (S Z))) (S (S Z))


Please enter a command:
{- BetaReduction ds_d2Mn -}

((\n_a20X -> case S (S Z) of
    Z -> n_a20X;
    (S m_a20Y) -> m_a20Y + S n_a20X
)) (S (S Z))


Please enter a command:
{- BetaReduction n_a20X -}

case S (S Z) of
    Z -> S (S Z);
    (S m_a20Y) -> m_a20Y + S (S (S Z))


Please enter a command:
{- Case Application -}

(S Z) (+) S (S (S Z))
\end{verbatim}
\caption{Derivation using the regular step command}
\label{fig:FR2exampleStep}
\end{figure}

Comparing the two examples shows that using the call command,
the user can skip 4 steps (1x delta reduction, 2x beta reduction, 1x case application)
without losing any valuable information.

\newpage
\subsection{FR 3}
The user can step through the reduction line by line.

\ \\
This requirement has been met as well.
It corresponds to the manual and interactive modes that are available when using the Substitution Stepper.
Through coloring the user can see which subterm is currently selected to be derived next and using the tab key the user can shift the selection to another subterm.
Additionally, the application indicates which redex is selected and how many redexes there are in total,
which makes it easier for the user to navigate the selection.

\begin{figure}[!ht]
\begin{verbatim}
$ stack exec -- corestepper-exe -m manual -v 1 -F ../test/Stepped.hs -f test
(fib) (S (S (S (S Z))))

(Subterm 1/1)


Please enter a command:
((fib) (S (S (S Z)))) + fib (S (S Z))

(Subterm 2/3)


Please enter a command:
((fib (S (S Z))) + fib (S Z)) (+) fib (S (S Z))

(Subterm 1/5)
\end{verbatim}
    \caption{Deriving the left occurrence of \texttt{fib} first.}
    \label{fig:FR3example1}
\end{figure}

\begin{figure}[!ht]
\begin{verbatim}
$ stack exec -- corestepper-exe -m manual -v 1 -F ../test/Stepped.hs -f test
(fib) (S (S (S (S Z))))

(Subterm 1/1)


Please enter a command:
(fib (S (S (S Z)))) + (fib) (S (S Z))

(Subterm 3/3)


Please enter a command:
(fib (S (S (S Z)))) (+) (fib (S Z)) + fib Z

(Subterm 1/5)
\end{verbatim}
    \caption{Deriving the right occurrence of \texttt{fib} first.}
    \label{fig:FR3example2}
\end{figure}

Figures \ref*{fig:FR3example1} and \ref*{fig:FR3example2} show the same term being derived.
However, in the second step, the user chooses two different redexes,
which leads to the two third steps looking different.

\subsection{FR 4}
The user can make the derivation run through to the end without requiring any interaction.

\ \\
The fourth FR is met as well.
There are two options for the user to let the derivation run automatically.

First, the user can use the automatic mode to step through the derivation without requiring any input.
The user can specify a strategy which is then used by the Substitution Stepper to determine the sequence of reductions.

\begin{figure}[!ht]
\begin{verbatim}
$ stack exec -- corestepper-exe -m automatic -F ../test/Stepped.hs -f testAdd
Derive (S Z) + S Z
--------Step 1---------
RewriteRule: DeltaReduction +

(((\ds_d4e0 n_a3sr -> case ds_d4e0 of
    Z -> n_a3sr;
    (S m_a3ss) -> m_a3ss + S n_a3sr
)) (S Z)) (S Z)

--------Step 2---------
RewriteRule: BetaReduction ds_d4e0

((\n_a3sr -> case S Z of
    Z -> n_a3sr;
    (S m_a3ss) -> m_a3ss + S n_a3sr
)) (S Z)
\end{verbatim}
\end{figure}
\begin{figure}[!ht]
\begin{verbatim}
--------Step 3---------
RewriteRule: BetaReduction n_a3sr

case S Z of
    Z -> S Z;
    (S m_a3ss) -> m_a3ss + S (S Z)


--------Step 4---------
RewriteRule: Case Application

Z + S (S Z)
\end{verbatim}
\end{figure}
\begin{figure}[!ht]
\begin{verbatim}
--------Step 5---------
RewriteRule: DeltaReduction +

(((\ds_d4e0 n_a3sr -> case ds_d4e0 of
    Z -> n_a3sr;
    (S m_a3ss) -> m_a3ss + S n_a3sr
)) Z) (S (S Z))

--------Step 6---------
RewriteRule: BetaReduction ds_d4e0

((\n_a3sr -> case Z of
    Z -> n_a3sr;
    (S m_a3ss) -> m_a3ss + S n_a3sr
)) (S (S Z))
\end{verbatim}
\end{figure}
\begin{figure}[!ht]
\begin{verbatim}
--------Step 7---------
RewriteRule: BetaReduction n_a3sr

case Z of
    Z -> S (S Z);
    (S m_a3ss) -> m_a3ss + S (S (S Z))


--------Step 8---------
RewriteRule: Case Application

S (S Z)
\end{verbatim}
    \caption{Deriving a term using the interactive mode.}
    \label{fig:FR4example1}
\end{figure}

\clearpage
When using the manual derivation mode,
the user also has the choice to make the derivation continue automatically.
This can be seen in figure \ref*{fig:FR4example2},
where after the first manual step the derivation is continued automatically.

\begin{figure}[!ht]
\begin{verbatim}
$ stack exec -- corestepper-exe -m manual -v 2 -F ../test/Stepped.hs -f testAdd
(S Z) (+) S Z

(Subterm 1/1)


Please enter a command:
{- Normalize (((\ds_d2MJ n_a21a -> case ds_d2MJ of
    Z -> n_a21a;
    (S m_a21b) -> m_a21b + S n_a21a
)) (S Z)) (S Z) -}


Z (+) S (S Z)

(Subterm 1/1)


Please enter a command:
\end{verbatim}
\end{figure}
\begin{figure}[!ht]
\begin{verbatim}
Derive Z + S (S Z)
--------Step 1---------
RewriteRule: DeltaReduction +

(((\ds_d2MJ n_a21a -> case ds_d2MJ of
    Z -> n_a21a;
    (S m_a21b) -> m_a21b + S n_a21a
)) Z) (S (S Z))

--------Step 2---------
RewriteRule: BetaReduction ds_d2MJ

((\n_a21a -> case Z of
    Z -> n_a21a;
    (S m_a21b) -> m_a21b + S n_a21a
)) (S (S Z))
\end{verbatim}
\end{figure}
\begin{figure}[!ht]
\begin{verbatim}
--------Step 3---------
RewriteRule: BetaReduction n_a21a

case Z of
    Z -> S (S Z);
    (S m_a21b) -> m_a21b + S (S (S Z))

--------Step 4---------
RewriteRule: Case Application

S (S Z)

Derivation Finished.
\end{verbatim}
    \caption{Deriving a term using manual mode and then executing the continue command.}
    \label{fig:FR4example2}
\end{figure}

\subsection{FR 5}
The tool supports a verbose variant, that indicates which function has been applied.

\ \\
This requirement has been met as well.
There are up to 4 verbosity levels, depending on the derivation mode.

The automatic mode has two verbosity levels that satisfy the requirement.
Verbosity level 1 does not show the justifications for the reduction that was applied,
while verbosity level 2 does.

The manual mode has four verbosity levels.
The first two levels skip certain intermediate derivation steps according to FR 2,
while the two last levels do not skip anything.
In addition to that,
the even verbosity levels show the justifications while the odd levels do not.

The interactive mode has two verbosity levels.
The justifications will always be shown,
but the step size is configurable.
Verbosity levels 1 and 2 skip certain derivation steps, just like in the manual derivation,
while verbosity levels 3 and 4 do not skip anything.
So effectively, there are only 2 levels, even though 4 can be specified.

The following 4 figures \ref*{fig:FR5verbosit1}, \ref*{fig:FR5verbosit2}, \ref*{fig:FR5verbosit3}, and \ref*{fig:FR5verbosit4} show the same term being derived in manual mode.
Only the first step is shown and each of the figures has a different verbosity level.

\begin{figure}[!ht]
\begin{verbatim}
$ stack exec -- corestepper-exe -m manual -F ../test/Stepped.hs -f test -v 1
(fib) (S (S (S (S Z))))

(Subterm 1/1)


Please enter a command:
(fib (S (S (S Z)))) (+) fib (S (S Z))

(Subterm 1/3)
\end{verbatim}
    \caption{A step using verbosity level 1. Some intermediate steps are skipped and there are no justifications.}
    \label{fig:FR5verbosit1}
\end{figure}

\begin{figure}[!ht]
\begin{verbatim}
$ stack exec -- corestepper-exe -m manual -F ../test/Stepped.hs -f test -v 2
(fib) (S (S (S (S Z))))

(Subterm 1/1)


Please enter a command:
{- Normalize ((\ds_d2Na -> case ds_d2Na of
    Z -> S Z;
    S ds_d2Ni Z -> S Z;
    S ds_d2Ni (S n_a219) -> (fib (S n_a219)) + fib n_a219
)) (S (S (S (S Z)))) -}


(fib (S (S (S Z)))) (+) fib (S (S Z))

(Subterm 1/3)
\end{verbatim}
    \caption{A step using verbosity level 2. Some intermediate steps are skipped and justifications are included.}
    \label{fig:FR5verbosit2}
\end{figure}

\begin{figure}[!ht]
\begin{verbatim}
$ stack exec -- corestepper-exe -m manual -F ../test/Stepped.hs -f test -v 3
(fib) (S (S (S (S Z))))

(Subterm 1/1)


Please enter a command:
((\ds_d2Na -> case ds_d2Na of
    Z -> S Z;
    S ds_d2Ni Z -> S Z;
    S ds_d2Ni (S n_a219) -> (fib (S n_a219)) + fib n_a219
)) (S (S (S (S Z))))

(Subterm 1/4)
\end{verbatim}
    \caption{A step using verbosity level 3. No steps are skipped and no justifications are displayed.}
    \label{fig:FR5verbosit3}
\end{figure}

\begin{figure}[!ht]
\begin{verbatim}
$ stack exec -- corestepper-exe -m manual -F ../test/Stepped.hs -f test -v 4
(fib) (S (S (S (S Z))))

(Subterm 1/1)


Please enter a command:
{- DeltaReduction fib -}


((\ds_d2Na -> case ds_d2Na of
    Z -> S Z;
    S ds_d2Ni Z -> S Z;
    S ds_d2Ni (S n_a219) -> (fib (S n_a219)) + fib n_a219
)) (S (S (S (S Z))))

(Subterm 1/4)
\end{verbatim}
    \caption{A step using verbosity level 4. No steps are skipped and justifications are displayed.}
    \label{fig:FR5verbosit4}
\end{figure}


\clearpage
\subsection{FR 6}
The user can import user-defined files and step through functions defined in these files.

\ \\
This requirement is partially met.
While files cannot be imported per se, a file containing a Haskell module can be specified as an argument to the Substitution Stepper.
However, it is not currently possible to specify multiple modules.
Due to this limitation,
the requirement is only considered met partially.


\clearpage
\section{Non-Functional Requirements}

\subsection{NFR 1}
The tool is usable for people with little experience in functional programming.
The derivation of a user-defined function takes max. one import command and one step command.
The function is defined in one file and depends only on other functions in the same file.
\ \\
The first non-functional requirement has been met.
Stepping a function automatically can be done very easily.
The only thing that needs to be done is to execute the executable and passing file path and function name like this:

\begin{figure}
    \begin{verbatim}
    stack exec -- corestepper-exe -F ../test/Stepped.hs -f test
    \end{verbatim}
    \caption{Executing the Substitution Stepper only requires passing two simple arguments}
\end{figure}

There is no further interaction required when using the automatic derivation,
so it is very simple to use.
There is no experience needed in order to use the application.

\subsection{NFR 2}
The tool provides an open API or CLI, to open it up to additional UIs.
\ \\
I would also say that the second NFR is met.
The different derivation modes can be accessed easily.

To use the Substitution Stepper with a different UI,
the UI needs to be added to the \texttt{Manual.hs} and \texttt{Interactive.hs} files in the \texttt{src/Derivation} folder.
The \texttt{Automatic.hs} file does not require any changes at all since it does not perform any user interaction.

\ \\
To use the automatic derivation with a different UI,
the \texttt{doAutomaticDerivation} function can be called from the new UI.
The printing can then be done via the \texttt{prettyDerivation} function.

\ \\
To use the manual derivation with a different UI there are two options.
The easier option would be to rewrite the \texttt{ManualUI} module in the \texttt{src/UI} folder.
The functions in there can be adjusted to the new UI for different displaying or prompting of terms and commands.

Should that not offer enough customizability,
another UI module could be written,
which could then be worked into the \texttt{Manual} module in the \texttt{src/Derivation} folder.
If it behaves similarly to the \texttt{ManualUI} module,
then the functions in the \texttt{Manual} module can just be changed to call the new UI rather than the old one.
Only if it is very different from the original UI in how it works would the replacement of the UI be a bit of a bigger task.

\ \\
Using the interactive derivation with a different UI can be achieved the same way as for the manual derivation but instead of modifying the \texttt{ManualUI} or \texttt{Manual} modules,
the \texttt{InteractiveUI} and \texttt{Interactive} modules need to be adjusted.

\ \\
Since in most cases,
it should not be too much work to adjust the UI as described above,
I would say that the requirement is met.

\section{Limitations}

\subsection{Prelude}
One of the biggest limitations of the Substitution Stepper currently is that the Prelude is not supported.
This includes datatypes like String and Int as well as all the functions that are defined in the prelude.
However, instances like Monad, Applicative, etc. can still be implemented and work.
Most of the prelude can be implemented easily,
which would also be a useful exercise for students of the Functional Programming class.
However, there are some types that cannot be implemented too easily,
like numerical datatypes and operations on them.

\ \\
The reason for this limitation has to do with the CoreStepper project,
which this application is based on.
It is currently not possible to compile expressions and functions that contain Prelude functionality to Core.

\subsection{IO}
It is currently unclear if the Substitution Stepper would be able to step impure IO functions.
This is partially due to the fact that there is no mechanism to perform the input or output operations with,
even if the limitation with the Prelude functionality would be resolved.

\section{Known Issues}

\subsection{LeftmostInnermost Strategy}
The leftmost innermost derivation strategy does not work as expected.
Due to the fact that function applications and operations are not saved infix in the Core representation,
the implementation for the leftmost innermost derivation Strategy is a bit faulty.

The implementation assumes that the arguments to a function/operation are a (in)direct child of the function/operation.
This is not the case in Core.
The second argument for the operation is a sibling of the operation in Core representation, not a child.
This causes the operation to be perceived as nested more deeply than the argument,
which can cause infinite recursion in the worst case when used with the automatic derivation mode.

This issue still needs to be looked into,
to get the ordering right.

\subsection{Errors when using Low Verbosity}
When using a low verbosity level (1 or 2) during manual or interactive derivation,
some terms containing let bindings might erroneously indicate that they cannot be derived, 
while in fact,
they should be derivable.

The same applies when using the call command on a term containing a let binding,
as in the background,
stepping at low verbosity and using the call command are the same operation.

The reason for this issue is not known yet and needs to be looked into.


\subsection{Normalization}
The normalization command when using manual derivation does not work as intended yet.
It does not always fully normalize the term and stops in seemingly random places.
The cause behind this issue is not clear yet and needs to be looked into.

\subsection{Invisible Beta Reductions}
Some functions in Core require types as arguments.
This is due to the explicit typing used in Core.
Some generic functions need to take types as arguments so the explicit typing works in later derivation steps aswell.
I did not want to display types,
because I think that they just clutter up the display.
This, however,
causes some lambdas to be applied to "invisible" arguments.
In fact,
they are applied to the types,
but the types are invisible to the user.

This might be confusing,
but it is necessary.
While one could remove the types from the terms,
there are lambdas that expect a type as an argument
and there is no way to distinguish these lambdas from other lambdas.
So if one were to remove the type but leave the lambda in there,
it would result in a lambda that cannot be further reduced,
since it is missing its argument.

Luckily,
this is only a problem when using the most verbose modes during stepping.
If a mode with lower verbosity is used,
this problem will not be noticeable at all.

