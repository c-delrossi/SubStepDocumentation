\chapter{Personal Notes}
Working on this thesis was very enjoyable for me.
I appreciated the chance to work on a bigger project using Haskell.
It was extremely interesting to use such a different language for a change,
rather than building something using a language that I had used a lot in the past.

However,
it was also a bit of a challenge to work with somewhat unfamiliar technology.
While the functional programming class gave me an interesting overview of the basics and a bit more advanced topics like monads,
there was still much to learn.
I learned a lot about functional programming,
like how to use monad transformers and also other types of monads that were not covered in the lecture.
It was very cool that I got to put all these concepts, that were new to me, to use in this project.
I have gained a big appreciation for all the different monads,
that offer really helpful functionalities,
and especially the Maybe monad,
which makes the handling of failures beautifully elegant.

Of course,
it was not all sunshine and rainbows.
At the start of the project,
the hurdle of understanding Core and all the new concepts seemed hard to overcome.
There were times when I was a bit frustrated,
especially before I got to write some more lines of code and progress seemed slow.
However, once I started coding and things seemed to fall into place,
it was a lot of fun and it was really great to see how the project progressed steadily.

It was also a bit challenging to work on this thesis all by myself,
but I also enjoyed the ability to work at my own pace and not have to coordinate with a partner.
In addition, I also had the support of the previous developers of the Substitution Stepper,
which was very helpful at times and which I was really grateful for.
Of course, I also appreciated the valuable input that I have gotten from my advisor,
helping me to create a usable application.

Overall I am happy with the product that I created,
even though there are still some rough edges and things that could be improved.
I enjoyed being able to work on a project that I think could be really useful to my fellow students
and many other people trying to get into functional programming.
